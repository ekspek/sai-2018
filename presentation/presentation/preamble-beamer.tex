\usepackage[version=3]{mhchem}

\usepackage[utf8]{inputenc}
\usepackage[T1]{fontenc}
\usepackage{polyglossia}
\setmainlanguage{portuges}
%\usepackage{microtype}
\usepackage{xcolor}
\usepackage{anyfontsize}
\usepackage{pdflscape}
\usepackage{bbding}

% -- Tipo de letra
%\usepackage[osf]{newpxtext}
%\usepackage{eulervm}
%\usepackage[scaled=1.05]{nimbusmononarrow}
\usepackage{fontspec}
\usepackage[sfdefault]{FiraSans} %% option 'sfdefault' activates Fira Sans as the default text font
\renewcommand*\oldstylenums[1]{{\firaoldstyle #1}}
\usepackage{FiraMono}
%\usepackage{newtxsf}

% -- Funções matemáticas extra
\usepackage{mathtools}
\usepackage{siunitx}

% -- Símbolos extra
\usepackage{amssymb}

\usepackage{textcomp}
\usepackage{gensymb}
\usepackage{cancel}

% -- Bibliografia
\usepackage[
	backend = biber,
	style = numeric,
	sorting = ynt
	]{biblatex}
\usepackage{fvextra}
\usepackage{csquotes}

% --  Definições de imagens
\usepackage{graphicx}
\graphicspath{{graphics/}}
\usepackage{caption}
\captionsetup{font=scriptsize}
\usepackage{subcaption}
\usepackage{afterpage}
\usepackage{tabularx}
%\usepackage[labelformat=empty]{caption}
\usepackage{multicol}
\usepackage{multirow}
\usepackage{booktabs}
\usepackage[export]{adjustbox}
\usepackage{caption}

% -- Desenhar circuitos elétricos e lógicos
\usepackage{tikz}
\usepackage{pgfplots}
\usetikzlibrary{arrows.meta,positioning,patterns}
\pgfplotsset{compat=1.5}
\pgfplotsset{table/search path = {data}}
\pgfplotsset{	/pgf/number format/use comma,}

\definecolor{ist-cyan}{cmyk}{1,0,0,0}
\definecolor{ist-gray}{cmyk}{0.2,0,0,0.8}

%\hypersetup{colorlinks,
%	linkcolor	= {white},
%	citecolor	= {ist-cyan},
%	urlcolor	= {ist-cyan}}

\usepackage{todonotes}

%%%%%%%%%%%%%%%%%%%%%%%%%%%%%%
%Cenas do beamer

\addbibresource{main.bib}
\addbibresource{graphics.bib}
\definecolor{istblue}{cmyk}{1,0,0,0} % ist blue)

% símbolo de "certinho"
\def\checkmark{\tikz\fill[scale=0.4](0,.35) -- (.25,0) -- (1,.7) -- (.25,.15) -- cycle;} 
\mode<presentation>
{
  \usetheme{Madrid}       % or try default, Darmstadt, Warsaw, ...
  \usecolortheme{orchid} % or try albatross, beaver, crane, ...
  \usecolortheme[named=istblue]{structure}
  \usefonttheme{default}    % or try default, structurebold, ...
  \setbeamertemplate{navigation symbols}{}
  \setbeamertemplate{caption}[numbered]
  \setbeamertemplate{itemize items}[circle] %ball,circle, square
}

\setbeamertemplate{bibliography item}{\insertbiblabel}
\setbeamertemplate{caption}{\raggedright\insertcaption\par}

\setbeamertemplate{caption}[numbered]
\setbeamerfont{institute}{size=\Large}
\setbeamerfont{date}{size=\small}
\setbeamerfont{author}{size=\small}

